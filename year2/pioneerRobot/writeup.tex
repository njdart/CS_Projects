\documentclass[10pt]{article}
\usepackage{a4wide}
\usepackage[english]{babel}
\usepackage{graphicx}
\usepackage{tabu}
\usepackage{textcomp}
\usepackage{fancyhdr}
\usepackage{lastpage}
\usepackage{titlesec}
\usepackage{lscape}
\usepackage{longtable}
\usepackage{color}
\usepackage{listings}
\usepackage{xkeyval}
\usepackage{hyperref}

\definecolor{mygreen}{rgb}{0,0.6,0}
\definecolor{mygray}{rgb}{0.5,0.5,0.5}
\definecolor{mymauve}{rgb}{0.58,0,0.82}

\lstset{ % Syntax highliughting for java
    backgroundcolor=\color{white},   % choose the background color; you must add \usepackage{color} or \usepackage{xcolor}
    basicstyle=\footnotesize,        % the size of the fonts that are used for the code
    breakatwhitespace=false,         % sets if automatic breaks should only happen at whitespace
    breaklines=true,                 % sets automatic line breaking
    captionpos=b,                    % sets the caption-position to bottom
    commentstyle=\color{mygreen},    % comment style
    deletekeywords={...},            % if you want to delete keywords from the given language
    escapeinside={\%*}{*)},          % if you want to add LaTeX within your code
    extendedchars=true,              % lets you use non-ASCII characters; for 8-bits encodings only, does not work with UTF-8
    frame=none,                      % adds a frame around the code
    keepspaces=true,                 % keeps spaces in text, useful for keeping indentation of code (possibly needs columns=flexible)
    keywordstyle=\color{blue},       % keyword style
    language=Octave,                 % the language of the code
    morekeywords={*,...},            % if you want to add more keywords to the set
    numbers=left,                    % where to put the line-numbers; possible values are (none, left, right)
    numbersep=5pt,                   % how far the line-numbers are from the code
    numberstyle=\tiny\color{mygray}, % the style that is used for the line-numbers
    rulecolor=\color{black},         % if not set, the frame-color may be changed on line-breaks within not-black text (e.g. comments (green here))
    showspaces=false,                % show spaces everywhere adding particular underscores; it overrides 'showstringspaces'
    showstringspaces=false,          % underline spaces within strings only
    showtabs=false,                  % show tabs within strings adding particular underscores
    stepnumber=5,                    % the step between two line-numbers. If it's 1, each line will be numbered
    stringstyle=\color{mymauve},     % string literal style
    tabsize=4,                       % sets default tabsize to 2 spaces
    title=\lstname                   % show the filename of files included with \lstinputlisting; also try caption instead of title
}
%%%%%%
%% Variables for version and release status
%% useage: \version
%%%%%%
\newcommand\module{CS26410}
\newcommand\moduleName{Introduction to Robotics}
\newcommand\authorText{Nicholas Dart}
\newcommand\authorUsername{nid21}
\newcommand\studentID{130057750}
\newcommand\assesser{Myra Wilson, Tom Blanchard}

%%%%%%
%% Alias
%%%%%%
%\newcommand{\sectionbreak}{\clearpage}    %% Allways start a section on a new page

\title{ \huge \module Assignment \\ \Large \moduleName}
\author{
    \vspace{100pt}
    \begin{tabular}{ r || l }
        Author          & \authorText (\authorUsername)\\
                        & \studentID \\
        Date Published  & \today \\
                        & \\
        Assessed By     & \assesser \\
        Department      & Computer Science \\
        Address         & Aberystwyth University \\
                        & Penglais Campas \\
                        & Ceredigion \\
                        & SY23 3DB \\
    \end{tabular} \\
    %Copyright \textcopyright Aberystwyth University 2014
    %get rid of the date on the titlepage
    \date{}
}

\pagestyle{fancy}
\fancyhf{}
\lhead{\module~Assignment}
\rhead{\authorText~-~\studentID}
\rfoot{Page \thepage \hspace{1pt} of \pageref{LastPage}}
\lfoot{Aberystwyth University - Computer Science}

\begin{document}
    \setcounter{page}{1}

    \maketitle
    \thispagestyle{empty}
    \clearpage

    %%\tableofcontents
    %%\clearpage

    \section{Experiments designed}
        The experiments I designed for this assignment were influenced by a relatively large amount of experience with robotics prior to this course. As such I know that simulations are often inadequate, even when they are physical constructs of the expected environment. 

        I decided to test the example code in the simulation and on the real robot; this code was simple and tried to stay equidistant from obstacles on either side of it's self, it used simple control and had no detection or avoidance of obstacles either directly in front or directly behind it.

        Along with the provided code, I made a simple program to go a set distance, turn 180\textdegree~and return the set distance to where it started. I also made a simple program to wander and stop if it was facing an obstacle within an arbitrary distance, during construction of this program I had no concept of distance within the simulation, so the distance I chose was arbitrary. I later adapted this program whilst working with the real robot to try and follow a person but not crash into them, this code was not tested on the simulator.

    \section{Differences between Simulation and Reality}
        From experience, there are many problems with a simulated environment; Firstly we were constantly having to recharge the robot, something which was not required in the simulation. I did notice that as the robots battery drained, the robot had less torque and moved slower, however I was not aiming for speed during out experiments so this was of little concern. There did not seem to be any effect on sensor readings due to low battery power. 

        Another difference we noticed was that the simulation was done in a 2D space, whereas reality is 3D; we noticed small objects, particularly chair legs and objects close to the robot body were not detected reliably or in some cases at all. The sonar sonar sensors also had a large amount of noise in their readings; this may because they have a very large range however I was only interested in the closest reading, so this only presented a problem when objects were all far away. 

        Turning with the robot presented many problems; I am unsure as to what mechanism was being used to calculate the current angle, however there seemed to be a large amount of accumulated error over time, suggesting a method such as a rotational accelerometer was used instead of a compass. this made tasks such as turning an exact amount (eg 180\textdegree) increasingly inaccurate.

    \section{Explanation of Differences}
        Most of the afore-mentioned differences between the simulation and reality are because; as the name suggests; it is a simulation of reality. Things like gravity, friction coefficient, battery power or age are assumed to be constant, which may be the cause for short duration experiments, however when transitioning from a simulation to reality there is always a noticeable difference between how the simulation has interpreter and presented reality and how it actually is. For example (from experience) motors usually have a preferred direction of travel, having a bias usually means that they will either have a greater maximum speed in a given direction than in the other, or their ``off'' state is not where one would logically expect (in the middle of fully on and fully off state). These are often factors either assumed to be accounted for in the simulation or missed entirely and can often throw off simulations.

    \section{Coping with Reality}
        Trying to simulate all aspects of the real world is unrealistic as effects can be varied and not obvious they are even affecting the simulated world. The purpose of a simulation is to gage generic responses to scenarios and provide convenient and safe testing of a task. However more realistic simulations can speed up development. A large amount of errors due to variations in hardware can not be compensated for except with ``magic values'', however algorithms such as PID can be used to compensate for errors in hardware, providing appropriate sensors are present to determine error over time. 

    \newpage
    \section{Code Used}
        \lstinputlisting[language=c++,basicstyle=\tiny]{example.cpp}
        \newpage
        \lstinputlisting[language=c++,basicstyle=\tiny]{simple.cpp}
        \newpage
        \lstinputlisting[language=c++,basicstyle=\tiny]{followXander.cpp}
\end{document}
